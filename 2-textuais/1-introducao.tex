\chapter{Introdução}
\label{cap:introducao}

Um compilador é uma ferramenta usada para compilar código-fonte de uma linguagem de alto nível para código de máquina. Esse processo é feito em várias fases. As três primeiras fases do processo de compilação podem ser definidas como análise léxica, análise sintática e análise semântica. Elas são chamadas coletivamente de \textit{front-end} do compilador \cite{mogensen2024introduction}. A fase de análise léxica gera \textit{tokens} usados pela fase de análise sintática para validar que a entrada segue a estrutura da gramática da linguagem alvo e gerar uma árvore sintática para ser usada pelas próximas fases da compilação \cite{thain2020introduction}.

O programa que realiza a análise sintática é chamado analisador sintático ou \textit{parser}. Os \textit{parsers} podem ser classificados em dois tipos. O primeiro tipo é o \textit{bottom-up} (ou ascendente) que funciona reduzindo os \textit{tokens} às produções da gramática. O segundo tipo é o \textit{top-down} (ou descendente) que segue o caminho oposto do \textit{parser bottom-up} tentando encontrar produções correspondentes à estrutura da \textit{string} de entrada comparando-a com as produções da gramática \cite{cooper2022engineering}.

A disciplina de compiladores está presente em muitas grades curriculares de cursos de ciência da computação pelo mundo. Dentro dessa disciplina são ensinados vários algoritmos de análise sintática. No entanto, aprender o funcionamento desses algoritmos é uma tarefa difícil para os alunos, assim como também é difícil para os professores ensinarem esse assunto \cite{sangal2018pavt}.

Tendo em mente a dificuldade no ensino sobre compiladores, foram criadas ferramentas de visualização de algoritmos(ou \textit{AV} do inglês \textit{Algorithm Visualization}). Essas ferramentas procuram ajudar no ensino de algoritmos que fazem parte das fases da compilação e, ao serem usadas com alunos, tiveram uma resposta positiva. Elas podem auxiliar na compreensão do conteúdo não só por meio da visualização do funcionamento dos algoritmos, mas também pelas respostas instantâneas, que os estudantes podem usar para validar resultados que são difíceis de construir manualmente. \cite{10.1145/3002136}.

O interesse pelo uso de \textit{AV}'s cresceu nos últimos anos. Com disso, também houve o surgimento de tecnologias de desenvolvimento \textit{web} modernas e os avanços na capacidade e qualidade dos gráficos de \textit{browsers}. Esses avanços permitiram um desenvolvimento \textit{web} mais fácil e a criação de várias aplicações baseadas inteiramente na \textit{web} \cite{effectiveav}. Dentre os avanços no desenvolvimento \textit{web} está a criação de \textit{frameworks web} que reutilizam componentes e implementam funcionalidades para acelerar a criação de aplicações \textit{web} \cite{uppal2022}. Um exemplo de \textit{framework web} é o \textit{Svelte}\footnote{https://svelte.dev/} que foi utilizado para o desenvolvimento da ferramenta apresentada nesse trabalho.

Embora esse tema tenha já tenha sido abordado em trabalho similares como o proposto por \textcite{pavt}, ainda há avanços que podem ser alcançados como será discutido na seção de trabalhos relacionados. Com isso, esse trabalho apresenta o desenvolvimento da ferramenta de visualização de analisadores sintáticos VANSI\footnote{https://vansi.netlify.app}.

\section{Objetivos}
O objetivo desse trabalho é desenvolver uma ferramenta que auxilie na compreensão do funcionamento da análise sintática e quais os processos necessários para obter a saída de cada passo de seus algoritmos. Esse trabalho tem os seguintes objetivos específicos:
\begin{itemize}[label=$\sbullet$]
    % \item Criar a ferramenta usando \textit{Svelte}\footnote{https://svelte.dev/}.
    \item Desenvolver uma forma de visualização para os algoritmos CLR, SLR e LL(1).
    \item Validar a ferramenta fazendo uma avaliação com alunos.
\end{itemize}

